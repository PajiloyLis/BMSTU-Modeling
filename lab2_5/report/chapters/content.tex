\chapter{Задание}

Разработать программу для моделирования информационного центра. Моделируемая система состоит из трех операторов и двух обрабатывающих аппаратов.

Заявки поступают через интервал времени $10 \pm 2$ минуты. Если все три имеющихся оператора заняты, клиенту отказывают в обслуживании.

Операторы имеют разную производительность и обеспечивают выполнение запроса $20 \pm 5$, $40 \pm 10$, $40 \pm 20$ минут. Клиенты занимают свободного оператора за пренебрежимо малое время.

Полученные запросы поступают в накопители. Запросы первого и второго операторов поступают в первый накопитель, третьего~---~во второй.

Запросы выбираются на обработку двумя обслуживающими аппаратами из соответствующих очередей. Время обработки в обслуживающих аппаратах составляет $15$ и $30$ минут соответственно.

Необходимо смоделировать процесс обработки $300$ запросов и определить вероятность отказа.

\chapter{Теоретическая часть}
\section{Концептуальная модель}
На рисунке~\ref{fig:2_1} представлена концептуальная модель моделируемой СМО.

\begin{figure}[H]
	\centering
	\includegraphics[scale=0.7]{./images/r1.png}
	\caption{Концептуальная модель СМО}
	\label{fig:2_1}
\end{figure}
\section{Модель в терминах СМО}
На рисунке~\ref{fig:2_2} представлена схема модели в терминах СМО.

\begin{figure}[H]
	\centering
	\includegraphics[width=0.7\linewidth]{./images/sheme.pdf}
	\caption{Модель в терминах СМО}
	\label{fig:2_2}
\end{figure}

\section{Вероятность отказа}

Вероятность отказа вычисляется по формуле:
\begin{equation*}
	P = \frac{n_1}{n_0 + n_1},
	\label{eq:2_1}
\end{equation*}
где $n_0$~---~количество поступивших заявок, $n_1$~---~количество заявок, получивших отказ.

\chapter{Практическая часть}
\section{Результаты работы программы}

На рисунке~\ref{fig:res1} приведен графический интерфейс и результат работы программы.
\begin{figure}[H]
    \centering
    \includegraphics[width=0.9\linewidth]{images/res.png}
    \caption{Результат работы программы}
    \label{fig:res1}
\end{figure}