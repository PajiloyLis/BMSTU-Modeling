\chapter{Задание}

Разработать программу для моделирования системы массового обслуживания при помощи принципа $\Delta t$ и событийного принципа протяжки времени. Графический интерфейс должен позволять выбрать распределения генератора заявок и обслуживающего аппарата, задать параметры этих распределений, временной шаг для принципа $\Delta t$, число генерируемых заявок, вероятность возврата заявки в очередь. Интерфейс должен отображать максимальную длину очереди, полученную в ходе моделирования обоими методами протяжки времени.

Выбор закона распределения компонентов осуществляется среди следующих вариантов:
\begin{itemize}
	\item равномерное;
	\item нормальное;
	\item экспоненциальное;
	\item Пуассона;
	\item Эрланга.
	
\end{itemize}

\chapter{Теоретическая часть}
\section{Принцип $\Delta t$}

Приницп $\Delta t$ заключается в определении состояния всех блоков системы в момент времени $t+\Delta t$ на основании их их состояния в момент времени $t$. 

Основным недостатком такого принципа~---~необходимость анализировать состояния всех блоков, даже тех, чье состояние за данный промежуток времени $\Delta t$ не изменилось. Кроме того, при недостаточно малом $\Delta t$ существует риск пропуска отдельных событий в системе, малый же шаг $\Delta t$ приводит к дополнительным затратам машинного времени на анализ моментов времени, в которые состояние системы или отдельных ее компонентов не поменялось по сравнению с предыдущими.

\section{Событийный принцип}

Событийный принцип основывается на том свойстве моделируемых систем, что состояния отдельных устройств изменяются в дискретные моменты времени. На основе этого утверждения состояния блоков анализируются только в те моменты времени, в которые в системе происходит некоторое событие. Момент наступления следующего события определяется минимальным значением времени наступления из списка будущих событий, представляющего собой совокупность будущих изменений каждого из блоков системы. 


\chapter{Практическая часть}
\section{Результаты работы программы}
Результаты работы программы представлены на рисунках~\ref{fig:res1},~\ref{fig:res2}
\begin{figure}[H]
    \centering
    \includegraphics[width=0.9\linewidth]{images/no_return.png}
    \caption{Результат работы программы при отсутствии возвратов заявок в очередь}
    \label{fig:res1}
\end{figure}
\begin{figure}[H]
    \centering
    \includegraphics[width=0.9\linewidth]{images/return.png}
    \caption{Результат работы программы при возврате заявок в очередь с вероятностью 0.5}
    \label{fig:res2}
\end{figure}