\chapter{Задание}

Разработать программу для построения графиков графиков функций распределения и плотности вероятности по заданным математическому ожиданию и дисперсии, для следующих распределений:
\begin{itemize}
    \item равномерное;
    \item Пуассона;
    \item экспоненциальное;
    \item нормальное;
    \item Эрланга.
\end{itemize}

Интерфейс должен позволять выбрать распрделение из списка, ввести значения математического ожидания и дисперсии.

\chapter{Теоретическая часть}
\section{Равномерное распределение}
Случайная величина $X$ имеет равномерное распределение на отрезке $[a,b]$, если она непрерывна с функцией плотности вида:
\begin{equation}
    f(x) = 
    \begin{cases}
        c, & x \in [a, b] \\
        0, & x \notin [a, b]
    \end{cases}
\end{equation}
где $c$~---~$const$, $c=\frac{1}{b-a}$.

Функция распределения случайной величины, распрделенной по равномерному закону имеет вид:
\begin{equation}
    F(x) = 
    \begin{cases}
        0, & x < a\\
        \frac{x-a}{b-a}, & x \in [a, b]\\
        1, & x > b
    \end{cases}
\end{equation}

Математическое ожидание и дисперсия случайной величины распрделенной по равномерному закону:
\begin{equation}
    M[X] = \frac{a+b}{2}
\end{equation}
\begin{equation}
    D[X] = \frac{(b - a)^{2}}{12}
\end{equation}

Параметр $a$ равномерного распределения называется коэффициентом сдвига, и имеет физический смысл точки отсчета некоторой шкалы измерения. Параметр $b$ называется коэффициентом масштаба и физически его значение может быть интерпретировано как выбор шкалы измерения.

\section{Распределение Пуассона}
Случайная величина $X$ распределена по закону Пуассона с параметром $\lambda > 0$, если она принимает значения $0,\ 1,\ 2,\ ...$ с вероятностью:
\begin{equation}
    \label{eq:poisson}
    P\{X=k\} = \frac{\lambda^k}{k!}e^{-\lambda}
\end{equation}

Поскольку распределение Пуассона является дискретным, для него не определена функция плотности, но определена функция вероятности, представляемая в виде~\ref{eq:poisson}. 

Функция распределения вероятности для случайной величины, подчиняющейся закону Пуассона, имеет вид:
\begin{equation}
    F(k) = \sum_{n=0}^{k}\frac{\lambda^n}{n!}e^{-\lambda}
\end{equation}

Математическое ожидание и дисперсия распределения Пуассона определяются как:
\begin{equation}
    M[X]=\lambda
\end{equation}
\begin{equation}
    D[X]=\lambda
\end{equation}

Параметр $\lambda$ представляет собой среднее число событий, происходящих в единицу времени или на единице пространства, т.~е. интенсивность.

\section{Экспоненциальное распределение}
Случайная величина $X$ имеет экспоненциальное распределение с параметром $\lambda > 0$, если $X$~---~ непрерывна и ее функция плотности имеет вид:
\begin{equation}
    f(x) = 
    \begin{cases}
        \lambda e^{-\lambda x}, & x \geq 0\\
        0, & x < 0<
    \end{cases}
\end{equation}

Функция распределения вероятности случайной величины, подчиняющейся экспоненциальному закону, имеет вид:
\begin{equation}
    F(x) = 1-e^{-\lambda x}
\end{equation}

Математическое ожидание и дисперсия случайной величины подчиняющейся экспоненциальному закону, имеет вид:
\begin{equation}
    M[X] = \frac{1}{\lambda}
\end{equation}
\begin{equation}
    D[X] = \frac{1}{\lambda^2}
\end{equation}

Аналогично распределению Пуассона параметр $\lambda$ представляет собой интенсивность наступления некоторых событий.

\section{Нормальное распределение}
Случайная величина $X$ распределена нормально с параметрами $\mu$ и $\sigma > 0$, если $X$ непрерывна и функция плотности имеет вид:
\begin{equation}
    f(x) = \frac{1}{\sqrt{2\pi}\sigma}e^{-\frac{(x-\mu)^2}{2\sigma^2}}
\end{equation}

Функция распределения случайной величины, подчиняющейся нормальному закону, называется функцией Лапласа и имеет вид:
\begin{equation}
    \Phi(x) = \frac{1}{\sqrt{2\pi}}\int_{-\infty}^{x}e^{-\frac{t^2}{2}}dt
\end{equation}

Математическое ожидание и дисперсия нормального распределения определяются как:
\begin{equation}
    M[X] = \mu
\end{equation}
\begin{equation}
    D[X] = \sigma ^2
\end{equation}

Параметры $\mu$ и $\sigma$ являются коэффициентами сдвига и масштаба соотвественно.

\section{Распределение Эрланга}
Случайная величина $X$ имеет распределение Эрланга с параметрами $k \in N$ и $\lambda > 0$, если $X$ непрерывна и функция плотности имеет вид:
\begin{equation}
    f(x) = 
    \begin{cases}
    \frac{\lambda^k x^{k-1}e^{-\lambda x}}{(k-1)!}, & x \geq 0\\
    0, & x<0
        
    \end{cases}
\end{equation} 

Функция распределения вероятности случайной величины, подчиняющейся закону Эрланга, имеет вид:
\begin{equation}
    F(x) = 1-\sum_{n=0}^{k-1}\frac{1}{n!}e^{-\lambda x}(\lambda x)^n
\end{equation}

Математическое ожидание и дисперсия случайной величины распределенной по закону Эрланга определяются как:
\begin{equation}
    M[X]=\frac{k}{\lambda}
\end{equation}
\begin{equation}
    D[X] = \frac{k}{\lambda^2}
\end{equation}

С точки зрения потока событий параметры распределения Эрланга можно интерпретировать следующим образом: $k$~---~количество произошедших событий (или номер события время наступления которого необходимо определить), $\lambda$~---~интенсивность потока событий.
\chapter{Практическая часть}
\section{Результаты работы программы}
Результаты работы программы представлены на рисунках~\ref{fig:res1},~\ref{fig:res2},~\ref{fig:res3},~\ref{fig:res4},~\ref{fig:res5}.
\begin{figure}[H]
    \centering
    \includegraphics[width=0.9\linewidth]{images/uniform.png}
    \caption{Равномерное распределение}
    \label{fig:res1}
\end{figure}
\begin{figure}[H]
    \centering
    \includegraphics[width=0.9\linewidth]{images/poisson.png}
    \caption{Распределение Пуассона}
    \label{fig:res2}
\end{figure}
\begin{figure}[H]
\centering
\includegraphics[width=0.9\linewidth]{images/exponential.png}
\caption{Экспоненциальное распределение}
\label{fig:res3}
\end{figure}
\begin{figure}[H]
\centering
\includegraphics[width=0.9\linewidth]{images/normal.png}
\caption{Нормальное распределение}
\label{fig:res4}
\end{figure}
\begin{figure}[H]
\centering
\includegraphics[width=0.9\linewidth]{images/erlang.png}
\caption{Распределение Эрланга}
\label{fig:res5}
\end{figure}