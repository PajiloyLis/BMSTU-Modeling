\chapter{Задание}

Разработать программу для моделирования ресторана быстрого обслуживания. Моделируемая система состоит из трех операторов и четырех обрабатывающих аппаратов.

Заявки поступают в очереди к операторам через интервал времени $1.5 \pm 0.5$ минуты. 

Заявка поступает в очередь минимальной длины. При этом проверяются все очереди. Если длина очереди равна максимально возможной ($l_{max} = 5$), то для оператора соответствующего данной очереди происходит отказ.

Операторы имеют одинаковую производительность и обеспечивают обработку заявки за $5 \pm 1$ минут. Клиенты занимают свободного оператора за пренебрежимо малое время. При этом операторы имеют предустановленную вероятность технического отказа $p_{tr}=0.1$.

Полученные запросы поступают в накопитель, откуда выбираются тремя обслуживающими аппаратами. Эти аппараты имеют различную производительность и обрабатывают заявки за $14 \pm 1$, $15\pm 1$ и $14 \pm 2$ минуты. Обработанные заявки поступают в накопитель последнего обслуживающего аппарата.

Последний обслуживающий аппарат выполняет обработку заявок за время $2 \pm 1$ минуты.

Необходимо смоделировать процесс обработки $300$ запросов и определить следующие параметры:
\begin{itemize}
	\item для каждого оператора:
	\begin{itemize}
		\item вероятность отказа;
		\item количество обработанных заявок;
		\item общее время работы;
		\item среднее время обработки заявки;
	\end{itemize}
	\item количество обработанных заявок всеми операторами;
	\item среднее время работы всех операторов;
	\item вероятность отказа по всем операторам;
	\item для каждого обслуживающего аппарата:
	\begin{itemize}
		\item количество обработанных заявок;
		\item общее время работы;
		\item среднее время обработки заявки;
	\end{itemize}
\end{itemize}

\chapter{Теоретическая часть}
\section{Концептуальная модель}
На рисунке~\ref{fig:2_1} представлена концептуальная модель моделируемой СМО.

\begin{figure}[H]
	\centering
	\includegraphics[scale=0.5]{./images/concept.pdf}
	\caption{Концептуальная модель СМО}
	\label{fig:2_1}
\end{figure}
\section{Модель в терминах СМО}
На рисунке~\ref{fig:2_2} представлена схема модели в терминах СМО.

\begin{figure}[H]
	\centering
	\includegraphics[width=0.7\linewidth]{./images/SMO_Scheme.pdf}
	\caption{Модель в терминах СМО}
	\label{fig:2_2}
\end{figure}

\section{Вероятность отказа}

Вероятность отказа вычисляется по формуле:
\begin{equation*}
	P = \frac{n_1}{n_0 + n_1},
	\label{eq:2_1}
\end{equation*}
где $n_0$~---~количество поступивших заявок, $n_1$~---~количество заявок, получивших отказ.

\chapter{Практическая часть}
\section{Результаты работы программы}

На рисунке~\ref{fig:res1} приведен графический интерфейс и результат работы программы.
\begin{figure}[H]
    \centering
    \includegraphics[width=0.9\linewidth]{images/res.png}
    \caption{Результат работы программы}
    \label{fig:res1}
\end{figure}